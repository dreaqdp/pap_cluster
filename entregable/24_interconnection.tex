\subsection{Xarxa d'interconnexió}

Ja que tenim un nombre relativament reduït d'unitats, i volem dedicar tantes unitats com 
sigui possible a nodes de còmput, volem, per tant reduir el nombre d'unitats que dediquem
als commutadors. Per aconseguir aquesta fita, necessitem que cada commutador tingui el màxim
nombre de ports per unitat.
Com a efecte col·lateral d'aquesta estratègia, ens veiem amb un sistema que pràcticament es
veu forçat a utilitzar una topologia de xarxa de tipus malla.

Pel que fa a distribuidors, ens vam centrar en Mallanoxi Technologies. A la pàgina web
\cite{mellanox-web} de la companyia, vam trobar commutaadors InfiniBand, de diferents preus
i velocitats (com mes ports mes velocitat). A la tala \ref{tab:intercon} hem recollit els 
diferents models candidats a ser utilitzats en el nostre sistema.

La quantitat de commutadors que utilitzem en el nostre sistema es un factor clau, que marca,
tant el preu, com el rendiment, d'aquest. D'altre banda, pero, afecta diferent als dos parametres,
anteriors en funció dels altres components. Per tant no podem determinar encara quina sera la
millor configuració. No obstant si que podem reduir el rang de configuracions, per tal de que
l'analisis sigui mes ame.

Ja que utilitzem una topologia de tipus malla completa, el nombre de conneccions $C$ a nodes de comput
que podra tennir la xarxa ve marcat per el nombre de ports $P$ del que disposa cada commutador aixi
com el nombre de commutadors $N$ que tingui el sistema. Es facil deduir que el nombre de conneccions
que podra haver-hi al sistema ve determinat per $C = N \times ( P - N + 1 ) = -N^2 + N(P+1)$
\begin{itemize}
  \item 
  \item Per el cas del model MSX6018F-1SFS, ja que tenim 18 ports, el nombre de nodes que
    podem connectar-hi, ve determinat
\end{itemize}

\begin{table}[H]
\begin{adjustwidth}{-.5in}{-.5in}  
    \begin{center}
        \centering
        \scalebox{1.0}{
        \begin{centering}
            \begin{tabular}{l||c|c|c}
                \hline
                \cellcolor[HTML]{FFFFFF}Model         & Ports & Speed (Gb/s) & Preu (\$) \\ \hline \hline \rowcolor[HTML]{EFEFEF}
                MSX6012F-2BFS\cite{mellanox_msx6012f-2bfs} & 12      & 56           &  9309.00  \\ 
                MSX6018F-1SFS\cite{mellanox_msx6018f-1sfs} & 18      & 56           & 14791.00  \\ \rowcolor[HTML]{EFEFEF} 
                MSB7800-ES2F\cite{mellanox_msb7800-es2f}  & 36      & 100          & 25633.00  \\ 
                MQM8700-HS2F\cite{mellanox_mqm8700-hs2f}  & 40      & 200          & 29629.00  \\ \hline
            \end{tabular}
        \end{centering}
        }
    \caption{Comparació entre les diferents configuracions dels nodes.}
    \label{tab:intercon}
    \end{center}
\end{adjustwidth}
\end{table}
