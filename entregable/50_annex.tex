\appendix
\section{Spreadsheet} \label{sec:annex}
Per poder tenir totes les dades unificades i a l'abast de tots d'una manera senzilla, vam decidir treballar emprant \textit{Google Spreadsheet}. 

En aquest annex explicarem la seva estructura per tal de facilitar la comprensió de dels fulls de càlcul que hem utilitzat per poder realitzar tots els estudis presentats al llarg d'aquest projecte. Aquest és accessible via: \url{https://docs.google.com/spreadsheets/d/1h09R-LO8-yKG6tomeKPeug\_olNzxpoOcf6uOx5kS5uo/edit?usp=sharing}.

En les següents subseccions explicarem cadascuna de les fulles que composen l'spreadsheet.

\subsection{CPU}
En aquesta full hem analitzat diferents CPUs. Hem buscat processadors de gamma servidors/HPC. 

Per cada processador hem anotat les seves característiques per tenir un resum ràpid de tots ells i per poder extraure mètriques per analitzar el seu rendiment. Cada fila conté un processador diferent, i cada columna les dades d'aquests.

Hem calculat els GFlops en un únic socket, ja que així podem tenir el rendiment independentment del tipus de placa base, és a dir, sense tenir en compte si la placa base és single, dual, quad... socket. 
Aquesta columna és la de GFlops/Socket (columna \texttt{I}). La següent columna, la \texttt{J}, conté la mètrica de GFlops/Watt per poder tenir la seva eficiència energètica. A continuació, la columna \texttt{K} és per calcular els GFlops/Dòllar i així saber com de bé invertim els diners.

Les columnes de la \texttt{L} a la \texttt{N} han sigut per poder normalitzar les mètriques exposades respecte la mitjana i poder representar-les totes en un únic gràfic.

Les dades de preu dels processadors ARM, files \texttt{2} i \texttt{3}, les vam extreure de pàgines que no acabaven de concretar el preu de cada model, sinó que indicaven un ranc de preus per la família \textit{Thunder X2}. Per tant, per calcular el preu del 2n processador, el 2n més potent dels ARM, vam dividir el rank de preus entre el nombre de CPUs de la família. Sabem que no és una manera molt fiable de tenir el preu, però no vam aconseguir trobar el seu cost exacte enlloc.

Les dades de preu dels processadors AMD es van extreure tots de \url{www.neweggbussiness.com}. Les d'Intel es van trobar a la mateixa web del fabricant, Intel.

\subsection{Motherboards}
En aquesta fulla hi ha les plaques base que vam buscar, conté les seves característiques. Tenim el preu només d'algunes d'elles, ja que no va ser possible trobar la resta.

Hi trobareu dual i single-socket, totes sent per la mateixa família de processador, ja que els dos processadors preseleccionats pertanyen a la mateixa.

A l'eix de les \texttt{X} tenim les característiques, i al de les \texttt{Y} les plaques base buscades. No s'han extret mètriques.

\subsection{Memòries}
En aquesta full hi ha les memòries seleccionades, fent un resum de les seves característiques.

A l'eix de les \texttt{X} tenim les característiques, i al de la \texttt{Y} les memòries buscades.

\subsection{Mellanox}
Aquesta fulla conté la informació relativa a la xarxa d'interconnexió. S'anomena \texttt{Mellanox} perquè tots els commutadors buscats són de l'empresa Mellanox.

A l'eix de les \texttt{X} tenim les característiques, i al de la \texttt{Y} els commutadors buscats.

La columnna \texttt{H} conté les màximes connexions que podem tenir, en funció del nombre de switches i dels ports que tinguin. 
La següent columna, la \texttt{I}, indica quant d'espai necessitaríem als racks per tenir el total de commutadors i el màxim de connexions.
La columna \texttt{J} conté el màxim de nodes que podríem tenir, tenint en compte la limitació dels dos racks, de les 84 U. 
Aquest càlcul també s'ha realitzat per saber el nombre de nodes d'1 U que podríem encabir, el qual és a la columna \texttt{K}.

\subsection{GPU}
Aquest full conté les targetes gràfiques estudiades amb les seves especificacions tècniques i mètriques de GFlops, GFlops/Watt i GFlops/Dòllar.

Una característica de les targetes gràfiques és que, en funció del tipus de dada que han de computar, tenen una major o menor performance. La mètrica dels GFlops ja ve donada pels fabricants, per tant hem hagut de calcular les altres. 

Les mètriques s'han calculat per Floating Point (FP) 64, 32 i 16.
Les FP64 ens interessen perquè és la mida usual dels \texttt{double} i així podem comprar equitativament el rendiment amb la dels processadors, que ha estat calculada en base a dades de 64 bits.
Les FP16 són interessant perquè en el món de l'IA (Intel·ligència Artificial) s'acostumen a utilitzar dades d'aquesta mida. Usualment no es necessita més precisió i, així, es poden enviar més quantitat de dades en un mateix espai de memòria que si fossin dades de 64 bits, això és important perquè el bottleneck habitual en utilitzar GPUs és la connexió entre el processador i la GPU (queda limitada a la velocitat del PCIe).

\subsection{Nodes}
En aquesta fulla trobem les característiques que pot tenir el node, el seu preu i mètriques de rendiment.

Es pren com a base la motherboard seleccionada per cada processador a l'apartat \ref{sec:motherboard}. En funció d'aquestes es calcula el cost que tindria no tenir cap GPU, tenir-ne una o tenir-ne el màxim que permet la placa base. S'ha tingut en compte les 2 GPUs que més TFlops donen.

Hi ha marcades amb fons verd les combinacions de placa base i GPUs que més GFlops i GFlops/Dòllar donen.

\subsection{Summary}
Aquest full conté un resum de quines han sigut les combinacions de components establertes: processador i motherboard, targeta gràfica i tipus de switch; aconseguien més TFlops o GFlops/Dòllar.

A l'eix de les \texttt{X} hi ha tan els processadors com les possibles targetes gràfiques, aquests són per ambdós processadors. A l'eix de les \texttt{Y} tenim el tipus de switch, i en quin paràmetre ens fixem per trobar la millor combinació (TFlops o GFlops/Dòllar màxim).

Per cada combinació es busca quina seria la seva configuració en nombre de nodes d'1 i 2 U, els TFlops totals, preu i GFlops/Dòllar. La fila anomenada \texttt{column} és per facilitar la cerca dels altres paràmetres, per a no complicar excessivament les fórmules de les altres cel·les.

Per cada combinació de processador i GPU es marca a les files de \texttt{TFlops Total} en color blau quina és la que té el valor màxim. Les files de \texttt{Price} tenen el mínim pressupost marcat amb color verd. Les files de \texttt{GFlops/\$} tenen el màxim en color taronja.


\subsection{Roofline}
En aquest full exposem en l'eix de les \texttt{X} tenim els components i les seves característiques, i en l'eix de les \texttt{Y} tenim els càlculs de bandwith i performance. Cal destacar que en aquest full hi ha dues taules en una; les de bandwith i la de performance i s'han de llegir per separat.

\subsection{Summary X nintendos [Y ports]} \label{sec:annex_summary}
Aquesta subsecció engloba les següents fulles:
\begin{itemize}
    \item Summary 6 nintendos
    \item Summary 7 nintendos
    \item Summary 2 nintendos 36 ports
    \item Summary 3 nintendos 36 ports
    \item Summary 2 nintendos 40 ports
    \item Summary 3 nintendos 40 ports
\end{itemize}

Durant la realització d'aquest treball vam fer la broma que costava pronunciar \textit{switches} i que per tant anomenaríem els commutadors \texttt{nintendos} (switch $>$ Nintendo Switch $>$ nintendo). Per aquest motiu, vam començar a anomenar les fulles amb aquests noms.
Actualment, l'spreadsheet és tan gran i tantes coses depenen les unes de les altres que si canviem el nom de les fulles no pot aplicar bé els canvis de noms i se'ns trenca tot. És per aquest motiu que hem decidit mantenir els noms amb \texttt{nintendo}.

Totes les fulles segueixen la mateixa estructura i idea, simplement canviant el tipus de commutador, o \textit{nintendo}. Aquest canvi implica una diferència en el nombre de nodes que es poden tenir.

A l'eix de les \texttt{X} trobem el nombre de nodes d'1 i 2 U. Els nodes d'1 U depenen del nombre de nodes de 2 U. Els d'1 U es calculen realitzant el mínim entre el nombre de connexions totals dels switches menys els nodes de 2 U, i l'espai màxim (84 U) menys l'espai que ocupen els nodes de 2 U i els commutadors; és a dir, el mínim entre les connexions disponibles i l'espai disponible. 

A l'eix de les \texttt{Y} trobem els dos possibles processadors per cadascun les 7 possibles targetes gràfiques. A més, per cada CPU calculem quants TFlops obtenim per part de tots els processadors que podem incorporar, fila \texttt{TFlops cpus}, i quants TFlops per node de 2 U podem obtenir per part de les gràfiques tenint en compte la limitació del 30\% comentada a la secció \ref{sec:intro}, fila \texttt{TFlops/node 2U}.

Per cada targeta gràfica s'ha calculat:
\begin{itemize}
    \item Fila \texttt{Max ideal gpu}\\
        Màxim nombre de GPUs que podem tenir per node de 2 U, simplement tenint en compte els TFlops que podem obtenir per part de GPUs i els TFlops de FP64 de la GPU en qüestió.

    \item Fila \texttt{TFlops gpus}\\
        TFlops reals que podem aconseguir per part de les targetes gràfiques, tenint en compte la limitació del nombre de GPUs que podem encabir en funció de la placa base.\\
        El màxim de la fila està marcat amb el fons de la cel·la de color blau.
        
    \item Fila \texttt{TFlops total}\\
        TFlops del clúster, suma dels TFlops de les CPUs i les GPUs. \\
        El màxim de la fila està marcat amb el fons de la cel·la de color blau.

    \item Fila \texttt{\% GFlops GPU}\\
        Percentatge de GFlops que provenen dels acceleradors, per tal de conèixer fins a quin punt ens ajustem a la limitació del 30\%.

    \item Fila \texttt{Price (\$)}\\
        Cost total del clúster. Inclou el cost dels processadors, plaques base, memòries RAM, targetes de xarxa, commutadors i targetes gràfiques. \\
        En el cas que el cost superi el pressupost màxim en dòllars, la cel·la es marca amb color vermell.

    \item Fila \texttt{GFlops/(\$)}\\
        GFlops per dòllar del clúster.\\
        El màxim de la fila està marcat amb el fons de la cel·la de color taronja.
\end{itemize}
