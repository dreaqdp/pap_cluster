\documentclass{article}

\usepackage[catalan]{babel}
\usepackage[utf8]{inputenc}
\usepackage{fancyhdr}
\usepackage{graphicx}

\pagestyle{fancy}
\fancyhf{}
\rhead{PAP}
\lhead{Designing our own cluster}
\cfoot{Pàgina \thepage}

\begin{document}

\makeatletter
\begin{titlepage}
\thispagestyle{empty}
\begin{center}
	\centering
	\vspace{1cm}
	{\scshape\Large Projectes de laboratori de PAP\par}
	\vspace{0.5cm}
	{\Large Curs 2019/20 (Quadrimestre de primavera)\par}
	\vspace{3cm}
	{\huge\bfseries Designing our own\par HPC-oriented cluster\par}
	\vspace{6cm}
  {\setstretch{0.25}
  {\Large \itshape Rafel-Albert Bros Esqueu\par Andrea Querol de Porras\par Joan Vinyals Ylla-Català\par Pablo Vizcaino Serrano\par}
  }
  \vspace{0.5cm}
  \vfill
	{\large Facultat d'Informàtica de Barcelona\par}
\end{center}
\clearpage
\end{titlepage}


%https://docs.google.com/spreadsheets/d/1h09R-LO8-yKG6tomeKPeug_olNzxpoOcf6uOx5kS5uo/edit?usp=sharing


%https://www.top500.org/news/cavium-releases-thunderx2-arm-processor/
%https://www.ecmwf.int/sites/default/files/elibrary/2018/18590-how-arms-entry-hpc-market-might-affect-meteorological-codes.pdf
%https://store.avantek.co.uk/avantek-64-core-cavium-thunderx2-arm-server-r281-t94.html
%https://store.avantek.co.uk/catalogsearch/result/?q=Thunderx2
%https://www.gigabyte.com/ARM-Server/Marvell-ThunderX2
%https://www.itjungle.com/2018/04/09/counting-the-cost-of-ibm-i-on-power9-entry-systems/
%https://en.wikichip.org/wiki/cavium/thunderx2/cn9980

%Power of CN9980
%https://www.servethehome.com/cavium-thunderx2-review-benchmarks-real-arm-server-option/

\section{Descripció}

La primera qüestió a plantejar-nos ha estat quin seria el processador meś adient. Per a prendre una decisió de la forma més encertada possible, hem cregut oportú elaborar una taula amb diverses possibilitats i comparar les característiques de totes elles. Els principals criteris valorats per a prendre aquesta elecció, tal com es pot veure a la figura \ref{chartCPUs}, han estat els GFlops/Socket, els GFlops/Watt i els GFlops/Dòllar (sent els GFlops normalitzats). 

\begin{figure}[h]
    \centering
    \includegraphics[width=\textwidth]{entregable/fitxers/chartCPU}
    \caption{Comparativa entre les diverses CPUs valorades.}
    \label{chartCPUs}
\end{figure}

Després d'estudiar els resultats, per tenir un major marge de maniobre de cara a decisions futures hem pre-seleccionat dos processadors: l'\textit{AMD EPYC 7702P} \cite{AMDE7702P} i l'\textit{AMD EPYC Rome 7502P} \cite{AMDER7502P}, el primer per mostrar un excel·lent equilibri entre les tres principals característiques estudiades, i el segon per oferir el GFlop més barat sense menystenir excessivament els altres dos aspectes principals.

Seguidament, continuant amb la dinàmica establerta, hem prosseguit a analitzar diversos candidats per a tots i cadascun dels possibles components del clúster. A més, certs elements són opcionals i cal estudiar la seva viabilitat, com per exemple les GPUs.  %No m'acaba de fer el pes aquest paràgraf per lligar.

%S'ha de seguir però tinc son, això és tot per avui amics.

\section{Components}

\section{Representació Roofline}

\begin{thebibliography}{}

\bibitem{AMDE7702P} AMD EPYC 7702P: \textit{Newegg Business}, \\
https://www.neweggbusiness.com/product/product.aspx?item=9b-19-113-583
\bibitem{AMDER7502P} AMD EPYC Rome 7502P: \textit{Newegg Business}, \\ https://www.neweggbusiness.com/product/product.aspx?item=9siv1dsa223\\241\&bri=9b-19-113-589

\end{thebibliography}

\end{document}
