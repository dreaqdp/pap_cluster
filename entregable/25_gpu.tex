\subsection{GPU}

Finalment, queda plantejar-nos l'últim element pel nostre clúster: les unitats de processament gràfic o GPUs. Un cop més, el procediment ha estat buscar diverses GPUs candidates i comparar les seves característiques bolcant-les en una taula, tal com es pot veure a la taula \ref{tab:gpu_specs}.


\begin{table}[H]
\begin{adjustwidth}{-1.0in}{-1.0in}  
    \begin{center}
    \centering
    \scalebox{0.8}{
\begin{tabular}{cc||c|c|c|c|c|c|c}
    \hline
    \multicolumn{2}{c||}{Model}                                                                                                 & \begin{tabular}[c]{@{}c@{}}\cellcolor[HTML]{FFFFFF}NVIDIA \\ Tesla \\ V100 \cite{gpu_v100}\end{tabular} & \begin{tabular}[c]{@{}c@{}}NVIDIA \\ RTX \\ 2080 TI \cite{gpu_rtx_2080}\end{tabular} & \begin{tabular}[c]{@{}c@{}}NVIDIA \\ Quadro RTX \\ 8000 \cite{gpu_rtx_8000}\end{tabular} & \begin{tabular}[c]{@{}c@{}}NVIDIA \\ Tesla \\ T4 \cite{gpu_tesla_t4}\end{tabular} & \begin{tabular}[c]{@{}c@{}}AMD Radeon \\ Pro WX \\ 8200\cite{gpu_pro_8200}\end{tabular} & \begin{tabular}[c]{@{}c@{}}AMD Radeon \\ Instinct \\ MI50 \cite{gpu_mi50}\end{tabular} & \begin{tabular}[c]{@{}c@{}}AMD Radeon \\ Instinct \\ MI25 \cite{gpu_mi25}\end{tabular} \\ \hline \hline
                                  & \cellcolor[HTML]{EFEFEF}Freq (MHz)                                                     & \cellcolor[HTML]{EFEFEF}1246                                    & \cellcolor[HTML]{EFEFEF}1350                                     & \cellcolor[HTML]{EFEFEF}1395                                         & \cellcolor[HTML]{EFEFEF}585                                   & \cellcolor[HTML]{EFEFEF}1200                                         & \cellcolor[HTML]{EFEFEF}1200                                           & \cellcolor[HTML]{EFEFEF}1400                                           \\ \hline
                                  & Type                                                                                   & hbm2                                                            & gddr6                                                            & gddr6                                                                & gddr6                                                         & hbm2                                                                 & hbm2                                                                   & hbm2                                                                   \\
                                  & \cellcolor[HTML]{EFEFEF}GB                                                             & \cellcolor[HTML]{EFEFEF}16 o 32                                 & \cellcolor[HTML]{EFEFEF}11                                       & \cellcolor[HTML]{EFEFEF}48                                           & \cellcolor[HTML]{EFEFEF}16                                    & \cellcolor[HTML]{EFEFEF}8                                            & \cellcolor[HTML]{EFEFEF}32                                             & \cellcolor[HTML]{EFEFEF}16                                             \\
\multirow{-3}{*}{Memòria}             & \begin{tabular}[c]{@{}c@{}}Bandwidth\\ (GB/s)\end{tabular}                             & 900                                                             & 616                                                              & 672                                                                  & 320                                                           & 512                                                                  & 1024                                                                   & 484                                                                    \\ \hline
                                  & \cellcolor[HTML]{EFEFEF}Interface                                                      & \cellcolor[HTML]{EFEFEF}PCIe 3.0                                & \cellcolor[HTML]{EFEFEF}PCIe 3.0                                 & \cellcolor[HTML]{EFEFEF}PCIe 3.0                                     & \cellcolor[HTML]{EFEFEF}PCIe 3.0                              & \cellcolor[HTML]{EFEFEF}PCIe 3.0                                     & \cellcolor[HTML]{EFEFEF}PCIe 3.0 \& 4.0                           & \cellcolor[HTML]{EFEFEF}PCIe3.0                                        \\
                                  & Lanes                                                                                  & 16                                                              & 16                                                               & 16                                                                   & 16                                                            & 16                                                                   & 16                                                                     & 16                                                                     \\
\multirow{-3}{*}{Interconnexió} & \cellcolor[HTML]{EFEFEF}\begin{tabular}[c]{@{}c@{}}Bandwidth \\ (GB/s)\end{tabular} & \cellcolor[HTML]{EFEFEF}32                                      & \cellcolor[HTML]{EFEFEF}32                                       & \cellcolor[HTML]{EFEFEF}32                                           & \cellcolor[HTML]{EFEFEF}32                                    & \cellcolor[HTML]{EFEFEF}32                                           & \cellcolor[HTML]{EFEFEF}32                                             & \cellcolor[HTML]{EFEFEF}32                                             \\ \hline
                                  & {Power (W)}                                                                              & 300                                                             & 250                                                              & 260                                                                  & 70                                                            & 230                                                                  & 300                                                                    & 300                                                                    \\ \hline
    & \cellcolor[HTML]{EFEFEF}Preu (\$)                                                     & \cellcolor[HTML]{EFEFEF}7,399.00 \cite{gpu_v100_buy}                                & \cellcolor[HTML]{EFEFEF}1,159.99 \cite{gpu_rtx_2080_buy}                                 & \cellcolor[HTML]{EFEFEF}4,902.00 \cite{gpu_rtx_8000_buy}                                     & \cellcolor[HTML]{EFEFEF}2,199 \cite{gpu_tesla_t4_buy}                                  & \cellcolor[HTML]{EFEFEF}969.99 \cite{gpu_pro_8200_buy}                                      & \cellcolor[HTML]{EFEFEF}4,438.00 \cite{gpu_mi50_buy}                                       & \cellcolor[HTML]{EFEFEF}9,156.65 \cite{gpu_mi25_buy}                                       \\ \hline
                                  &  GFlops                                                                                 & 7,235.58                                                        & 420.20                                                           & 509.80                                                               & 254.40                                                        & 672.00                                                               & 6,865.92                                                               & 768.00                                                                 \\
                                  & \cellcolor[HTML]{EFEFEF}GFlops/W                                                       & \cellcolor[HTML]{EFEFEF}24.12                                   & \cellcolor[HTML]{EFEFEF}1.68                                     & \cellcolor[HTML]{EFEFEF}1.96                                         & \cellcolor[HTML]{EFEFEF}3.63                                  & \cellcolor[HTML]{EFEFEF}2.92                                         & \cellcolor[HTML]{EFEFEF}22.89                                          & \cellcolor[HTML]{EFEFEF}2.56                                           \\
\multirow{-3}{*}{FP64}            & GFlops/\$                                                                              & 0.98                                                            & 0.36                                                             & 0.10                                                                 & 0.12                                                          & 0.69                                                                 & 1.55                                                                   & 0.08                                                                   \\ \hline
                                  & \cellcolor[HTML]{EFEFEF}GFlops                                                         & \cellcolor[HTML]{EFEFEF}14,469.12                               & \cellcolor[HTML]{EFEFEF}13,772.80                                & \cellcolor[HTML]{EFEFEF}16,701.44                                    & \cellcolor[HTML]{EFEFEF}8,336.38                              & \cellcolor[HTML]{EFEFEF}11,008.00                                    & \cellcolor[HTML]{EFEFEF}13,731.84                                      & \cellcolor[HTML]{EFEFEF}12,584.96                                      \\
                                  & GFlops/W                                                                               & 48.23                                                           & 55.09                                                            & 64.24                                                                & 119.09                                                        & 47.86                                                                & 45.77                                                                  & 41.95                                                                  \\
\multirow{-3}{*}{FP32}            & \cellcolor[HTML]{EFEFEF}GFlops/\$                                                      & \cellcolor[HTML]{EFEFEF}1.96                                    & \cellcolor[HTML]{EFEFEF}11.87                                    & \cellcolor[HTML]{EFEFEF}3.41                                         & \cellcolor[HTML]{EFEFEF}3.79                                  & \cellcolor[HTML]{EFEFEF}11.35                                        & \cellcolor[HTML]{EFEFEF}3.09                                           & \cellcolor[HTML]{EFEFEF}1.37                                           \\ \hline
                                  & GFlops                                                                                 & 28938.24                                                        & 27545.60                                                         & 33402.88                                                             & 66693.12                                                      & 2201.60                                                              & 27463.68                                                               & 25169.92                                                               \\
                                  & \cellcolor[HTML]{EFEFEF}GFlops/W                                                       & \cellcolor[HTML]{EFEFEF}96.46                                   & \cellcolor[HTML]{EFEFEF}110.18                                   & \cellcolor[HTML]{EFEFEF}128.47                                       & \cellcolor[HTML]{EFEFEF}952.76                                & \cellcolor[HTML]{EFEFEF}9.57                                         & \cellcolor[HTML]{EFEFEF}91.55                                          & \cellcolor[HTML]{EFEFEF}83.90                                          \\
\multirow{-3}{*}{FP16}            & GFlops/\$                                                                              & 3.91                                                            & 23.75                                                            & 6.81                                                                 & 30.33                                                         & 2.27                                                                 & 6.19                                                                   & 2.75 \\ \hline
\end{tabular}
    }
\end{center}
\end{adjustwidth}
    \caption{Característiques i rendiment de les targetes gràfiques estudiades.}
    \label{tab:gpu_specs}
\end{table}


En el mercat actual hi ha principalment dos desenvolupadors de targetes gràfiques: NVIDIA i AMD. La recerca de GPUs s'ha centrat en models d'aquestes dues marques. Tradicionalment, NVIDIA ha tingut la principal quota de mercat en l'entorn de la supercomputació, però últimament AMD està desenvolupant productes per aquest nínxol que estan competint frec a frec amb els de NVIDIA.

S'han buscat dos tipus de models per cada desenvolupador: un amb molta potència de càlcul i l'altre amb uns GFlops inferiors. L'explicació és que així ens podíem plantejar dos escenaris, un per cada tipus de model. 
Els models amb molts GFlops els vam buscar pel cas de tenir només uns pocs nodes amb moltes GPUs, és a dir, concentrant la potència de càlcul.
L'altre tipus de model, amb menys GFlops per unitat, es va escollir pel suposat escenari de tenir la potència de càlcul de la GPU en més nodes, o fins i tot en tots.

Aquests dos escenaris presenten avantatges i inconvenients. En el cas de tenir pocs nodes amb molts GFlops concentrats:
\begin{itemize}
    \item Avantatges:
    \begin{itemize}
        \item Menys nodes de 2U i més d'1U. Això permetria tenir més nuclis en el clúster i per tant més nodes per escalar aplicacions o executar diverses aplicacions sense haver d'esperar recursos disponibles.
    \end{itemize}
    \item Inconvenients:
    \begin{itemize}
        \item Els usuaris hauran de compartir nodes per poder accedir a les GPUs. 
        \item En el cas que una aplicació utilitzi MPI i CUDA/OpenCL, es veurà limitada l'escalabilitat en nombre de ranks MPI perquè si tots llencen workloads de gràfiques hauran d'estar en nodes amb gràfiques disponibles.
    \end{itemize}
\end{itemize}

En l'escenari de tenir més nodes amb targetes gràfiques, els avantatges i desavantatges passen a ser els inconvenients i les avantatges del cas anterior, respectivament.

Una diferència important entre NVIDIA i AMD és el model de programació amb el que es pot programar. 
NVIDIA té un model de programació propietari, anomenat CUDA \cite{gpu_cuda}. És relativament senzill de programar amb aquest model. Per això, moltes aplicacions que estan optimitzades per a utilitzar gràfiques s'han programat amb CUDA i no amb OpenCL.

En canvi, OpenCL és un model de programació obert per sistemes heterogenis \cite{gpu_opencl}. La diferència amb CUDA és que no només pot llençar workloads de feina a gràfiques, sinó que a qualsevol element de comput del sistema com altres cores o FPGAs (Field-Programmable Gate Array). A més, és més complicat de programar
Aquest model de programació paral·lela no és tant senzill d'utilitzar com CUDA i és menys utilitzat.

Les targetes gràfiques d'NVIDIA poden executar tant CUDA com OpenCL, però la performance amb OpenCL és inferior a la que es pot obtenir amb CUDA. En canvi, les GPUs d'AMD es poden programar amb OpenCL i no amb CUDA. 
Això representa una diferencia important, ja que aplicacions que només estiguin programades amb CUDA no podran fer ús de GPUs AMD.

D'aquest component no farem cap preselecció perquè dependrà de diversos factors quina targeta gràfica escollim. Aquests factors són: 
\begin{itemize}
    \item Processador: en funció del processador tindrem més GFlops o menys, per tant els GFlops procedents d'acceleradors seran majors o menors, sempre mantenint-nos dins del límit de que màxim el 30\% dels GFlops del clúster poden provenir d'acceleradors.
    \item Xarxa d'interconnexió: la quantitat de nodes que es puguin connectar ens permetrà tenir més o menys nodes de 2U, és a dir, nodes amb GPUs. Per exemple, si tenim una xarxa amb pocs ports i queden Us al rack sense connexions, es pot optar per utilizar nodes que ocupin més espai (nodes amb gràfiques), aprofitant al màxim les connexions i l'espai.
\end{itemize}


