\section{Introducció}\label{sec:intro}

En aquest projecte ens proposem dissenyar un clúster d'HPC (\textit{High Performance Computing}) obeint les restriccions establertes en la documentació. Aquestes condicions seran explicades en detall exposant, com no podria ser d'altra manera, les problemàtiques causades de cara a les decisions que caldrà prendre a l'hora d'escollir els diversos components del clúster.

\subsection{Restriccions del disseny}
En primer lloc, hi ha una restricció molt clara i sense marge per dubtes: el cost màxim per adquirir tots els components del clúster ha de ser d'un milió i mig d'euros. Això requerirà tenir ben present el preu de tots els elements i procurar no superar el límit.

L'espai també queda definit: disposem de dos armaris, cadascun d'ells de 42 U i, en definitiva, 84 U totals. Aquells nodes que contenen una gràfica ocuparan 2 U, mentre que aquells que no en tinguin ocuparan 1 U. Cal entendre que a més commutadors major serà el nombre de nodes que podem connectar a la xarxa però, paral·lelament, menor serà el nombre d'unitats que quedaran disponibles pels nodes de computació. Un cop s'hagi establert el nombre d'unitats i nodes disponibles, s'haurà de cercar la millor combinació possible entre nodes amb GPUs i nodes lliures d'elles.

Pel que fa a GPUs, el nostre disseny ve delimitat per la condició que fins a un màxim del 30\% dels flops del sistema poden provenir de nodes de computació accelerats, és a dir, de nodes amb GPUs. Tanmateix, el nombre màxim de GPUs dependrà de la placa base escollida, que al seu torn depèn del processador que haguem escollit. En definitiva: les decisions s'hauran de prendre amb peus de plom, sempre mantenint la possibilitat de fer un pas enrere i modificar qualsevol elecció prèvia.

Finalment, no existeix cap restricció específica pel que fa al consum energètic, però evidentment és important intentar maximitzar l'eficiència energètica del clúster. 

%Sou benvinguts de retocar el que us surti dels pebrots.

